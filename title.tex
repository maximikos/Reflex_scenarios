% Inspired by title template from ShareLaTeX Learn; Gubert Farnsworth & John Doe
% Edited by Max Koslowski, template from Jon Arnt Kårstad, NTNU IMT

\begin{titlepage}
\vbox{ }
\vbox{ }
\begin{center}
% Upper part of the page
\includegraphics[width=0.40\textwidth]{Images/NTNU_logo.png}\\[1.5cm]
\textsc{\LARGE Industrial Ecology Programme}\\[1.5cm]
\textsc{\large EP8114 Term paper}\\[0.5cm]
\textsc{\large \emph{by} Max Koslowski}\\[1cm]
\vbox{ }

% Title
\HRule \\[0.8cm]
{ \centering \huge \bfseries
A criticism of conventional and\\[0.6cm]
a plea for \textit{real} scenarios
}\\[0.8cm]
\HRule \\[1cm]

%Abstract
\renewcommand{\abstractname}{}
\begin{abstract}
\textbf{Abstract$\vert$} Socio-ecological systems have long been not accounted for in their full complexity. Wicked problems such as climate change gave rise to the field of sustainability science to do so, with prospective analyses having become an important piece in the repertoire. In this essay, however, I contend that current sustainability scenarios are insufficient in that they are dominated by one model family, highly sensitive to model traits, and biasing the wider sustainability scenario discourse and further modelling efforts. To resolve these issues, I plead for \textit{real} scenarios that are characterised as \textit{complete}, \textit{exhaustive}, and \textit{deeply integrated}. Through the adaptation of these, I claim that the scenario possibility space would be explored more completely, thus allowing for more nuanced responses to wicked problems.
\end{abstract}


% Author
%\large
%\emph{Author:}\\
%Maximilian Koslowski
\vfill

% Bottom of the page
{\large December, 2022}
\end{center}
\end{titlepage}