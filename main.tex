\documentclass{article}
\usepackage{packages}
\begin{document}

\import{./}{title}

\frontmatter
\tableofcontents

\mainmatter
\linenumbers

{\huge{\multiTitle}}

\section*{ABSTRACT}
Socio-ecological systems have long been not accounted for in their full complexity. Wicked problems such as climate change gave rise to the field of sustainability science to do so, with prospective analyses having become an important piece in the repertoire. In this essay, however, we contend that current sustainability scenarios are insufficient in that they are dominated by one model family, highly sensitive to model traits, and biasing the wider sustainability scenario discourse and further modelling efforts. To resolve these issues, we plead for \textit{real} scenarios that are characterised as \textit{complete}, \textit{exhaustive}, and \textit{deeply integrated}. Through the adaptation of these, we claim that the scenario possibility space would be explored more completely, thus allowing for more nuanced responses to wicked problems.

%I guess this LCA overview paper may be relevant \url{https://www.frontiersin.org/articles/10.3389/frsus.2022.629653/full}

% Check out this PNAS feature on sustainability transitions of production-consumption systems \url{https://www.pnas.org/topic/551}


\label{intro}

\section{Introduction}
A very short introduction to the challenges we address.

What are scenarios? In the context of sustainability
science, integrated scenarios may be thought of as
coherent and plausible stories, told in words and
numbers, about the possible co-evolutionary pathways
of combined human and environmental systems. (Swart)
It is becoming increasingly clear that such pathways need to recognise the dependence of the human on the environmental systems. (Pauliuk Hertwich)

Swart paper as guiding element

Maybe useful to add that there's (seemingly) no agreement on the definition of the term "scenario" \url{https://onlinelibrary.wiley.com/doi/pdfdirect/10.1002/ffo2.3}

As for a classification of scenario, I might just fall back on the typology by Börjeson, which is also described in Bishop et al \url{https://www.researchgate.net/profile/Andy-Hines/publication/228623754_The_current_state_of_scenario_development_An_overview_of_techniques/links/0f31753c036cf1a857000000/The-current-state-of-scenario-development-An-overview-of-techniques.pdf}

Other typologies and classifications are provided/summarised by Amer et al. \url{https://www.sciencedirect.com/science/article/pii/S0016328712001978}, or earlier by van Notten et al. \url{https://www.sciencedirect.com/science/article/pii/S0016328702000903}

This article by Elsawah et al. seems to be very comprehensive when it comes to the scenario generation process \url{https://www.sciencedirect.com/science/article/pii/S0048969720319069}

This paper by Elsawah et al. identifies eight grand challenges in socio-environmental systems modelling \url{https://sesmo.org/article/view/16226/15789}. Maybe useful in combination with Swart paper to narrow down our contribution? (As well as the review paper by Wiebe et al. that also mention difficulties arising from scenario adaptation and customisation \url{https://www.annualreviews.org/doi/full/10.1146/annurev-environ-102017-030109}). Maybe one could argue that the feedback loop of assumptions into assumptions presens a separate grand challenge (admittedly a not obvious one which is ssimultaneously just emerging now when more and more (sustainability) modelling communities jump onto the train of prospective modelling), lying right at the interdisciplinary section of primary scenario developers and users/customisers?

This overview article by Guivarch mentions that "too ittle attention is given to the implication of plausible alternative model functional forms", thus asking a question about model comparability and the definition of plausibility embedded there \url{https://www.sciencedirect.com/science/article/pii/S1364815217303936}. The article is mainly however an overview of recent additions to scenario techniques.

Maier et al. on multiple plausible futures and its relation to deep uncertainty \url{https://www.sciencedirect.com/science/article/pii/S1364815216300780}

Possibly a good review on scenario techniques and the topic of black swans; they also develop a new technique \url{https://www.sciencedirect.com/science/article/pii/S0016328722000398}

And this article appears to classify existing typologies \url{https://www.sciencedirect.com/science/article/pii/S0040162519306924}

An overview of what makes socio-environmental modelling more useful to policymakers \url{https://besjournals.onlinelibrary.wiley.com/doi/pdfdirect/10.1002/pan3.10207}

I think that the essence of my time-dependent plausibility critique might be captured, at least partially, by "Moreover, the scenario users are reflexive in their activities: prepared not only to change their route and destination, but also to remodel their ship and its crew." as Wilkinson write \url{https://iopscience.iop.org/article/10.1088/1748-9326/3/4/045017/pdf}

Selin et al. provide a good overview of recent progress in dynamic sustainability scenario modelling, writing that many old challenges are started to get addressed \url{https://www.pnas.org/doi/abs/10.1073/pnas.2216656120}

Donges et al. distinguish between different taxa (=subsystems) of world-Earth systems. I suppose one might argue that scenario models should be designed in such a way that the do not violate any fundamental law in any taxon or at any interaction between taxa. \url{https://esd.copernicus.org/articles/12/1115/2021/} At the same time, some laws might of course still be unknown (known and unknown unknowns), such that models/assumptions might violate them unknowingly. Also, model abstraction might inadvertently violate such laws (when some dynamics are not represented sufficiently).

Aside from climate mitigation scenarios, there's now a big push from (IAM-)modellers to also account for nature more comprehensively: the nature futures framework "emerged to inspire the development of nature and people positive, diverse values-integrated, and multiscale scenarios" \url{https://www.sciencedirect.com/science/article/pii/S095937802300047X}. That framework overlaps well with Donges' world-Earth system taxonomy and earlier concept of techno- and ecosphere or also Stefan's paper on socio-economic metabolism.

One thing that should also be clear (and highlighted) is that there might be conflicting (scenario-)targets! Hence, the discussion on possibility space and plausibility might have to be widened, simply so that more comprehensive explorations of multiple objectives can be examined. (the nature futures scenarios might be a good example for that when put next to most of the climate mitigation scenario literature). I should add some studies highlighting for instance the conflicting nature of the SDGs (going basically back to wicked problems). Moreover, some very popular scenario sets that are sometimes used for "general" sustainability modelling have actually specific foci, such as the RCPs/SSPs on climate mitigation and adaptation. Also this better literacy of scenario applicability is necessary to improve.

Perhaps distinguish in secondary models between those doing basically just accounting and those that capture dynamics (i.e. essentially the attributional-consequential divide). The call for better model/scenario/assumption understanding goes to both, although at differing degrees of "severity". Also, the accounting models might rely on sometimes strong assumptions, depending on how abstract they are. Some approaches can be very problematic (at least conceptually), such as prospective attributional LCA where data from multiple time series is taken to analyse an impact for a particular point in time (e.g. taking time-ignorant ecoinvent data and combining it with time-step specific IAM data as well as "dependent" (i.e. dynamic) foreground data.)

Key points of article:
. warn against the circular reference of assumptions (embedded in data, models, and scenarios); this also includes the point on researcher integrity
. argue that the criterion of plausibility needs to be time-dependent so that unknown unknowns might be better captured
. models might have to improve to be aligned with basic criteria (e.g. follow thermodynamic laws)
. argue for better scenario applicability literacy (understand the implications of using specific sets of scenarios, i.e. the assumptions embedded in input data, model mechanisms, narratives, and output data)

\begin{refsection}

\section{A brief history of sustainability scenario modelling}

Economics and ecology have surely more in common than their etymological origin. And yet, while humanity's immediate dependence on its natural surroundings lies in plain sight, economics gradually reduced in its own considerations the latter to an abstract banality. From these segmented externalities regarded by Pigouvian taxes \parencites{pigou_2017}[see also][]{baumol_1972}, Coase's \parencite*{coase_1960} social costs, and the like -- and despite earlier fundamental insights by \textcite{carlowitz_1713}, \textcite{jevons_1865}, \textcite{arrhenius_1896}, and their respective contemporaries -- it was a long way until socio-ecological systems were again contemplated holistically, as was the case in the day of Aristotle. The seed for this imminent transition came not only from within economics, nor from ecology, but also included fields like environmental literature and philosophy; one may think of the works by Waldo Emerson, Henry David Thoreau, John Muir, Aldo Leopold, and Arne Næss.

Starting with generic musings by \textcite{daly_1968,isard_1968,cumberland_1966}, \textcite{ayres_kneese_1969} eventually presented in a seminal study how nature-society interactions could be acknowledged more systematically. One can only speculate about their motives of doing so in a neo-classical Walras-Cassel general equilibrium model and not in one more agnostic to economic schools of thought. Yet, before long, interest in this kind of academic exercise spiked rapidly. This was driven by crucial observations like those of Rachel \textcite{carson_1962}, new conjectures on the presumable roots of ecological crises \parencite{white_1967}, and powerful ideas such as the Gaia hypothesis \parencite{lovelock_1972,lovelock_1974}, the concept of a spaceship Earth \parencite{boulding_1966}, and the $I=PAT$ equation \parencite{ehrlich_1972,commoner_1972,chertow_2001}.

Parallel to important contributions in economics on the attribution of current environmental repercussions \parencite[see for example][]{leontief_1970,leontief_1972}, controversial neo-Malthusian concerns à la \textit{Tragedy of the Commons} \parencite{hardin_1968,ostrom_1999,mildenberger_2019} paved the way for the emerging field of world modelling, pioneered by the early activities of the Club of Rome. Different from siloed scientific approaches before, this new field had at its focus the holistic analysis of complex systems such as those which human-nature interactions comprise. Although coarse in detail, initial models like World3 used in \textit{The Limits to Growth} \parencite{meadows_1972} were groundbreaking. Soon after, more refined ones were developed to address the \textit{World Problematique}. The United Nations were instrumental in furthering this work \parencite{fontela_2004}, where some models emphasised the economic aspects (in the tradition of meso-economic and general equilibrium models, e.g. \textcite{leontief_1974}) whereas others focused on interactions within a system of systems (as in systems dynamics, e.g. \textcite{mesarovic_1974}).

With publications such as \textit{Only One Earth} \parencite{ward_1972} and \textit{Our Common Future} \parencite*[WCED,][]{brundtland_1987} documenting the rising necessity to reconsider then-current socio-economic trajectories, the need for quantitative models kept on increasing. Already back then, the important idea of inner and outer limits under various spatio-temporal considerations arose which is popularised today under the name of an inter- and intragenerationally equitable doughnut economy where human needs are satisfied within planetary boundaries; a concept that is in some aspects also approximated by the ideas of deep ecology and strong sustainability. Thus, it is not surprising that the scope of many computational models expanded, both in breadth and in detail. In the following years, various interdisciplinary models (often under the umbrella term of energy-economy models) emerged and first steps in the direction of truly integrated models were undertaken. The meaning of integration goes here in two ways: vertical integration in the sense of linking causes and consequences of environmental issues along their causal chain, and horizontal integration denoting the breadth of causal links as well as cross-topic representations \parencite{parson_1997}. This level of integration has certainly stood in contrast to the granularity of the system representation, and continues to do so today \parencite{krey_2014}.

Although the relevance of a changing climate was acknowledged in the scientific community long before James Hansen's testimony to the US Congress, the first proper integrated assessment model (IAM), RAINS \parencite[developed during the 1980s; for an overview see][]{alcamo_1991}, dealt with acid rain. Yet, RAINS served as a blueprint for similar models that would address climate change and other topics. Such models became relatively quickly part of decision-making processes; as was the case with, for example, ASF \parencite{lashof_1989} and IMAGE \parencite{rotmans_1990} in the first report by the Intergovernmental Panel on Climate Change \parencite*[IPCC,][]{ipcc_1991}.

When speaking of IAMs, one must distinguish between cost-benefit and process-based models: the former are in direct lineage of Coase's concept of social cost and most often are highly aggregated models, whereas most process-based models emerged from the literature on energy modelling and only few were elevated from meso-economic models.\footnotemark{} \textcite{nikas2019,hafner2020} provide useful, detailed classifications of IAMs.

Proctor on heterodox IAMs \url{https://link.springer.com/article/10.1007/s43253-023-00098-7}

A comprehensive comparative evaluation of process-based IAMs is here \url{https://link.springer.com/article/10.1007/s10584-021-03099-9}.

An overview of the IAMs and their historial role at the policy interface \url{https://doi.org/10.1016/j.gloenvcha.2020.102191}

\footnotetext{Cost-benefit models such as PAGE \parencite{yumashev_2019}, FUND \parencite{waldhoff_2014}, and DICE/RICE \parencite{nordhaus_2017} are also termed policy optimisation models, whereas process-based models such as IMAGE \parencite{stehfest_2014}, MESSAGE \parencite{huppmann_2019}, and REMIND \parencite{baumstark_2021} are often referred to as policy evaluation models. The solution concept -- optimisation or simulation -- is, however, not the eponym for these two classes of models; it is rather their use that is the reason for their naming. Cost-benefit models are typically employed to identify the \textit{optimal} balance (based on some criterion) between costs and benefits of policies and impacts. In contrast, process-based models are used to evaluate socio-economic and environmental consequences of actions and impacts; depending on their level of integration and system representation, process-based models are sometimes also referred to as energy-economy-environment (E3) or E4 (E3 + engineering) models. Some models, such as WITCH \parencite{bosetti_2006,emmerling_2016}, can be employed for both policy optimisation and evaluation. In fact, cost-benefit and process-based IAMs may be calibrated to the results of the other \parencite{vanden_2020}, as well as against more complex impact models \parencite{moss_2010}. Both types of IAMs have repeatedly been subject to fundamental criticism \parencite[see for example][]{stern_2016,weyant_2017,keen_2021,gambhir_2019,farmer_2015,pindyck_2017}.}

The spawn of IAMs coincided with sustainability concerns seeping into other scientific disciplines. Earth system models, for example, expanded from modelling only biogeochemical processes of the Earth system to also account for human drivers of changes \parencite{flato_2011}.\footnotemark{} And while it seemed for a while that sociology was slow to pick up on it \parencite{passerini_1998}, it is now rather the case that its potent contribution to sustainability discourses is overlooked \parencite{longo_2021}. Meanwhile, in one branch of engineering, nature-society interactions started to be examined on a rather granular level while still maintaining a system's perspective \parencite{frosch_1989,ausubel_1992}. Characterised by its ideal of cyclic material flows and holistic views \parencite{jelinski_1992}, this new field of industrial ecology thus anticipated the concept of a circular economy \parencite{kalmykova_2018,merli_2018} and, together with other approaches firmly rooted in the respect for thermodynamic principles \parencite{haberl_2019}, the scientific paradigm of socio-economic metabolism \parencite{pauliuk_2015}. An emphasis on thermodynamic principles and roots in physiocratic thinking were also crucial for the emergence of ecological economics, a branch of economics that started growing after, and was moulded by the likes of Robert Ayres, Kenneth Boulding, and Nicholas Georgescu-Roegen \parencite{cleveland_1999}.

\footnotetext{The literature knows many attempts of coupling these models with IAMs. Although primarily \textit{simple} climate and Earth system models such as MAGICC \parencite{meinshauen_2011,meinshausen_2020} have actually been integrated into IAMs \parencite{vanVuuren_2011}, couplings of more complex models have been conducted as well and are expected to become more common \parencite{vanVuuren_2012}.}

Thus, one may observe today a large heterogeneity of models and model families used to analyse socio-ecological systems. These differ not only in their origins and traditions, but also in terms of the aspects they focus on and their underlying principles. For example, IAMs often do not explicitly represent society's biophysical basis, while this is essential to industrial ecology approaches \parencite{pauliuk_2017}. And while industrial ecology and ecological economics share assumptions and considerations, their detail of analysis often differs \parencite{kronenberg_2006}. In addition, one can identify even within model families large differences, such as regarding solution concepts in IAMs or the scale of analysis in industrial ecology.

While some of these differences are not unproblematic, for instance regarding the (dis)respect for thermodynamic laws, one could argue that despite their individual flaws such a large variety of approaches is required to adequately understand socio-ecological systems. Even more so given that in the short time span of the anthropocene \parencite{crutzen_2006}, we have moved quickly from only feeding on nature's free gift \parencite{karsten_1987} to grossly altering our surroundings on all levels in a full world \parencite{daly_2005}. Reports by the IPCC \parencite[e.g.][]{ipcc_2022}, the intergovernmental science-policy platform on biodiversity and ecosystem services \parencite*[e.g. IPBES,][]{ipbes_2019}, and the international resource panel \parencite[e.g.][]{unep_2016} have repeatedly documented humanity's voracity and its consequences. And yet, despite substantially transgressing Earth's biophysical boundaries and getting close to tipping points of all sorts \parencite{ceballos_2015,steffen_2015, steffen_2018}, we are failing on a large scale to achieve social targets \parencite{fanning_2022}.


\noindent\fbox{%
    \parbox{\textwidth}{%
        An overview of different representations of the future, based on \url{https://www.emerald.com/insight/content/doi/10.1108/FS-01-2021-0020/full/html}
    }%
}


It is therefore imperative for sustainability science and the associated disciplines to not only understand socio-ecological systems in hindsight \parencite{kates_2001} but also to provide prospective guidance by outlining potential futures \parencite{swart_2004}. This is underlined by both the precautionary principle\footnotemark{} as well as humanity's ancient wish to foresee the future. Although predictions of such complex systems as the human and natural systems are practically impossible, other ways of addressing the future have been devised \parencite{polasky_2011}, with scenario-modelling being well-established in sustainability science. While scenario modelling is the ultimate purpose of IAMs, it has rather slowly been picked up in other domains of sustainability assessment. The IAM community has therefore, and because of their models' comprehensive integration, an advantage as compared to other approaches. Scenario sets originating from the IAM community are highly authoritative and form the basis for global decision-making with regard to climate and other environmental change. Such scenario sets include in particular the representative concentration pathways \parencite[RCPs;][]{vanvuuren_2011_rcp} and the shared socio-economic pathways \parencite[SSPs;][]{riahi_2017,oneill_2014} in the context of climate change. Comprehensive scenario sets for the wider sustainability perspective are less common \parencite{vansoest_2019}, but include notably the sustainable development pathways developed by The World in 2050 initiative \parencite{twi_2018} and those by the SHAPE project \parencite[e.g.][]{soergel_2021}. In addition, modelling approaches from other domains of sustainability science often adopt the assumptions, input parameters, or scenario outputs from these authoritative sources. Examples include the studies by \textcite{deetman_2018,beltran_2020,schandl_2020}. Alternatively, modellers from other fields might devise their own scenarios or rely only partially on larger scenario sets \parencite[for example][]{pauliuk_2013}.

\footnotetext{It should be noted, though, that there are of course different (value-laden) perceptions and definitions of the precautionary principle \parencite{gardiner_2006}. Add \url{https://link.springer.com/article/10.1007/s11406-022-00582-0} and \url{https://www.tandfonline.com/doi/abs/10.1080/10807039991289185} and \url{https://www.tandfonline.com/doi/full/10.1080/21550085.2013.844569}}

In what follows (section \ref{main_part}), we contend that current sustainability scenario-modelling efforts are insufficient in that they are: i) dominated by one model family from which its shortcomings descend to other approaches (or alternatively, if sector-specific scenarios are crafted with other model families, that those might be inconsistent); ii) highly sensitive to model assumptions, structure, and parameterisation; and iii), because of this incomplete exploration of the possibility space, biasing the wider scenario discourse and further scenario modelling efforts. To resolve these issues, we suggest to move beyond conventional efforts and over to \textit{real} scenario-modelling. With that, we mean that each scenario set should be: a) \textit{complete}, and b) \textit{exhaustive}. In addition, c) we argue that both scenario sets and respective models should enhance \textit{deep} integration. What these terms imply is discussed in section \ref{discussion}.

\section{Scenario elements}
Assumptions, beliefs, ideas, emotions, values, norms...

Observations, expectations, imaginaries, narratives \url{https://journals.sagepub.com/doi/full/10.1177/00380385221138010#fn2-00380385221138010}

\noindent\fbox{%
    \parbox{\textwidth}{%
        The quick brown fox jumps right over the lazy dog. the quick brown fox jumps right over the lazy dog. the quick brown fox jumps right over the lazy dog. the quick brown fox jumps right over the lazy dog. the quick brown fox jumps right over the lazy dog. the quick brown fox jumps right over the lazy dog. the quick brown fox jumps right over the lazy dog. the quick brown fox jumps right over the lazy dog.
    }%
}

\subsection{A jungle of scenario qualifiers}
Possibility, probability, likeliness, plausibility, feasibility, desirability, consistency, contingency.

\subsection{Limiting criteria}
Explain why plausibility is a better criterion than the too-vast possibility and the conditional (because based on past and present trends into the unknown future) probability

Plausibility as an epistemic device, scenarios as perception devices (uruena). He also argues that plausibility requires an additional identifier/criterion.

Plausibility used in the scenario creation process and as a criterion for evaluating a scenario. (Uruena describes this as methodological-delimiting and anticipatory-enabling)

At the same time, plausibility is subjective. Would pre-cautionary approaches admonish us to think BIG and thus consider rather possibility?
At the same time, some have argued that applying precautionary approaches requires the identification of plausible threats...
Tipping points are of course of more uncertain nature, but other climate impacts and other environmental impacts more generally are more certain

\section{Possibility and solution spaces}

Scenarios are more than continuations of past legacy.

Future path dependency depends on past, present, and future decisions. Future path dependency does not continue right from present path dependency - the time in between matters and may feature creative and other disruptions.

Someone's plausible scenario output might be someone else's implausible scenario input (and vice versa)

Talk about black swans as well as grey swans / dragon kingsand grey rhinos. (How does it actually relate to the IPCC's definition of vulnerability, risk etc?)
(And with that also talk about regime changes, (social + natural) tipping points etc.)

\label{main_part}
\section{An unfortunate situation}

Scenarios are generally characterised as qualitative or quantitative manifestations of plausible and internally consistent narratives about the future. The antecedents of today's scenario-modelling exercises lie in war games, from where the use of scenarios spread to strategic planning in business and other organisations \parencite{bradfield_2005,schoemaker_1993}.\footnotemark{} With environmental concerns growing during the 1960s-70s, and emphasised by institutions like the Club of Rome and the United Nations, long-term scenario-modelling started to be also used for illuminating potential futures of socio-ecological systems.

A perhaps good resource for background and history on scenario-planning? \url{https://www.tandfonline.com/doi/full/10.1080/00076791.2020.1844667}

\footnotetext{Herman Kahn in the US and Gaston Berger in France were presumably the two most influential figures in these developments.}

Great strides have been made since these pioneering endeavours. And it seems that still more has to be done: While some of the problems addressed by earlier efforts appeared to be only one-dimensional and thus could be dealt with comparatively straightforwardly, we are currently experiencing polycrises in our increasingly volatile, uncertain, complex, and ambiguous world. Some of these crises (individually or combined), may be considered as wicked \parencite{rittel_1973, termeer_2019,lonngren_2021} or even super-wicked problems, such as climate change \parencite{levin_2012}. These are vastly more complex and difficult to solve than other problems that seemed insurmountable at first, such as the ostensible 1894 great horse-manure crisis or the issue of acid rain in late-20$^{th}$ century Europe. These current problems demand therefore more sophisticated approaches, both regarding problem understanding (i.e. modelling) and solving (i.e. policy responses). 

Today's sustainability scenario-modelling is largely dominated by the IAM community, their models, and the scenarios that they develop. While the development process for the most authoritative global change scenarios has evolved from moving only sequentially along the causal chain to a more parsimonious, parallel process \parencite{moss_2010} and ultimately combining scenario sets in a matrix architecture \parencite{vanvuuren_2014},\footnotemark{} SSPs rely just like RCPs (and often other frameworks, too) on marker scenarios. That is, for each of a set of agreed upon narratives there is one representative scenario implementation chosen. Non-marker scenarios can then be understood as alternative scenario interpretations \parencite{riahi_2017}. The marker scenarios are, however, the authoritative ones and are exclusively implementations made with major IAMs such as AIM \parencite{fujimori_2017}, GCAM \parencite{calvin_2017}, and MESSAGE \parencite{fricko_20171}. When such scenarios are then taken over by other assessment methods, it is usually only the marker scenarios that are employed, thus introducing a bias towards the bigger IAMs (and thus towards large modelling teams that might also receive comparatively more funding). Also across different scenario sets, there is a bias towards larger IAMs: among all 1,686 vetted pathways explored in the IPCC Sixth Assessment Report database using a pool of more than 20 models, about one third was computed using MESSAGE and REMIND \parencite{gambhir_2022,byers_2022}.

\footnotetext{Although many goals were achieved with this scenario architecture, the common sentiment among respective modellers is that further improvements must be made, such as the integration of societal conditions \parencite{oneill_2020}. Others have raised their voices concerning the plausibility of some scenario combinations, such as SSP3-RCP7.0 or RCP5-SSP8.5 \parencite{pielke_2022}.}

Differentiate between scenario adaptation and/or customisation and ignorant use of numbers. This review could be useful \url{https://www.annualreviews.org/doi/full/10.1146/annurev-environ-102017-030109}

On the other hand, these major IAMs have some of the most comprehensive socio-ecological system representations available. Other scenario-modelling approaches that focus only on a subsystem (e.g. a sector, or in terms of a partial equilibrium) might yield inconsistent modelling outcomes and scenario sets. The reason for this is simply that feedback effects (including rebound effects) might not be accounted for, thus potentially resulting in inconsistent dynamics of the system. In addition, insights from such exercises may be hard to compare to the outcomes from more comprehensive (although perhaps coarser) assessments.

Among IAMs, there are great differences in terms of model structures, input assumptions, and model outcomes \parencite{krey_2014,krey_2019,hiroto_2020,keppo_2021}. It may thus happen that one ends up comparing apples and oranges. For example, when a model (say, one relying on intertemporal optimisation) is fundamentally different from another one (e.g. a myopic systems dynamics model), and when their data parameterisations of the same aspects differ, too, one cannot expect to obtain like interpretations for a common exercise.\footnotemark{} Although the issue has already been recognised \parencite{giarola_2021}, there may continue to be cases where modelling outcomes are more sensitive to the choice of IAM than to the input parameterisation \parencite{sognnaes_2021}. At the same time, the case of similar models yielding different results because of only different parameterisations is hardly surprising either. On an even more fundamental level, IAMs might diverge in their worldviews, with many among the cost-optimizing frameworks following neoclassical assumptions. Well-known problematic aspects following from these worldviews often include the application of discount rates \parencite{emmerling_2019}, non-coverage  of capital trade (because of the otherwise occurring Lucas paradox \parencite{lucas_1990,keppo_2021}), a disregard for emerging behaviours \parencite{farmer_2015}, or unrealistic behavioural assumptions \parencite{asefi_2021}. Another issue is certainly when input or output parameters from one model (which might be based on neoclassical assumptions) are uncritically adopted by other modelling efforts that might have entirely different (and thus inconsistent) assumptions (e.g. post-Keynesian) - or where even the parameter sets are contradictory for some variables. It is even more problematic when results from these secondary efforts might find their way back again into the first set of models, thus creating a wild mix in terms of assumptions and parameterisations.\footnotemark{}

\footnotetext{On that note, it need be mentioned that the selective use of some scenarios and their (mis)interpretation is a topic in itself which has been addressed multiple times already \parencite{pielke_2021}, e.g. regarding the use of RCP8.5 as a baseline scenario \parencite{hausfather_2020}.}

\footnotetext{Importantly, some of the input assumptions of IAMs may be too coarse to use as an input for more disaggregated secondary modelling. It should be noted, however, that there also exist multi-sectoral IAMs of higher resolution; they do rely on different assumptions and principles, though \parencite{lefevre_2022}.}

These and other issues are the reasons for why it has been argued that IAMs do not explore the full possibility space \parencite{keppo_2021,mccollum_2020, gambhir_2022}. Assumptions integral to individual models then become inevitably implicit in scenario sets developed with these models. Since explicit and implicit value choices are made in the construction of both the IAMs and their scenario sets (and the underlying narratives, respectively), care must be applied when interpreting the outcomes of such efforts \parencite[even if it is only reference scenarios; see e.g.][]{grant_2020} - and even more so when adopting them for further (secondary) analysis with models from other domains. Given the relevance of global change scenarios for policy-making, one moves thus quickly from questions of intrinsic to extrinsic ethics, meaning that the wider society might be affected by modellers' judgements \parencite{beck_2016},\footnotemark{} for instance in terms of implicit justice perceptions \parencite{rubiano_2022}. One might even go so far as to allege a bias towards Western value sets, given that most of the IAMs are being developed in respective countries which are then also predominantly employed in multi-model comparison studies \parencite{duan_2019}.

\footnotetext{Given that wicked problems such as those addressed with IAMs cannot be "tamed" with only a reductionist approach \parencite[nor is it morally justifiable to do so;][]{churchman_1967}, especially not under the precautionary principle, ethical values come inevitably into play when widening the analysis' subject and the respective comprehensive modelling. This is typically the case when social dynamics are part of the problem -- as is common for wicked problems.}

As of now, IAMs are our best shot to tackle large-scale (or even wicked) problems comprehensively, but they fail on different levels. Especially the incomplete (and sometimes inadequate; cf. neoclassical assumptions) coverage of socio-economic dimensions in these models creates a bias towards irrational (although some economists might call it rational for exactly that) futures. Such model assumptions and biases are perpetuated and disseminated through respective scenarios to secondary modelling efforts and to the wider discourse on topics of global change. It is therefore that we argue that we need to move beyond these conventional scenario-modelling efforts and towards \textit{real} ones.

%%
%\import{./Sections/}{Discussion}
\label{discussion}
\section{...and a plea to better it}

With \textit{real} we do not mean realistic scenarios (nor even plausible ones, for that matter). Rather, we have seen great transformations in human history before \parencite{fischer_2014} so that we believe that scenario-modelling exercises should account for possible (again, not necessarily plausible from the current viewpoint), disruptive developments in the future.\footnotemark{} Support for this view comes also from within the IAM community. For example, \textcite{mccollum_2020} argue that IAMs should more vehemently explore extreme scenarios so as to account for transient (i.e. low-probability) events (e.g. sudden and widespread recessions), disruptive drivers (e.g. disintegration of political alliances), and unexpected outcomes (e.g. technological change). \textcite{gambhir_2022} point out that current IAMs do not allow for such a more complete exhaustion of the possibility space. Of course, a full exploration of the scenario space is virtually impossible, but we would nevertheless urge for a \textit{fuller} exploration than what is currently done. This relies on three points that we shall briefly explain in the following.

\footnotetext{Even if one disregards the great transitions of past centuries and millennia, one only has to consider the fact that current scenario models of global change explore the future to the year 2100 - if we just look back at what the world was like 80 years or so ago, still in the middle of WWII, one might realise that drastic changes have happened that were hard to foresee back then. Hence, more or less linear extrapolations of current conditions and thinking are grossly inappropriate, thus essentially ruling out all approaches that envision a future strongly based on the past. This is the case for practically all approaches based on statistical learning methods and for narratives that stick too closely to past experiences.}

First, each individual scenario must be \textit{complete}. With that we mean that a scenario must fully and transparently represent the entire system, not only a subsystem. This does not mean that every model employed necessarily has to be capable of a quantitative full system coverage; rather, the model must respect fundamental principles such as thermodynamic laws and produce a scenario that is in so far consistent and coherent as that feedback effects are sufficiently accounted for and that no contradictions are prevalent, both regarding its implementation and the narrative that it is based on.

Secondly, each scenario set must be \textit{exhaustive}. That is, a sufficiently large set of scenarios must be constructed which includes not only those that are thinkable given the current conditions but also those that are considered as (too) extreme. This relates to both more exploratory parameterisations of common narratives as well as the compilation of more out-of-the-ordinary narratives. A relatively mild version of the latter includes current post-growth considerations \parencite{hickel_2021}. An \textit{exhaustive} scenario set certainly requires also to have a wider range of models producing the individual (\textit{complete}!) scenarios (and thus perhaps to give up the concept of marker scenarios), such that more alternative worldviews and model philosophies are covered and find their way more easily into secondary modelling and policy discourses.\footnotemark{} We have seen in the past that transient events caused by great disruptions under fundamental uncertainty have happened. In future scenarios we have to account for them, too, via all means possible. If that is done under assignment of probabilities or not is secondary.

\footnotetext{This means to also expand from the over-reliance on current modelling frameworks, like the cost-optimisation ones (e.g. MESSAGE) or the ones based on systems dynamics (e.g. IMAGE) in their current form, to a more open landscape where alternative approaches based on e.g. agent-based modelling are sufficiently well represented, too. This would allow to tell entirely different stories than is now the case \parencite{proctor2023}. We do not argue, though, that the current major IAM modelling groups are solely to be blamed for the current situation; it is rather an unfortunate course of events (due to selective scenario applications and misinterpretations of individual scenarios and their implicit and explicit assumptions) that has put too much emphasis on only a limited set of models with limited worldviews and the thus produced scenarios.}

In addition, we argue that sustainability scenario assessments have to not only increase their vertical and horizontal integration (while not downgrading the level of representation to inappropriate levels), but also to enhance \textit{deep} integration. With that we mean both the integration of IAMs with other modelling approaches -- in the sense of a multi-model ecology \parencite{bollinger_2015} so as to improve the consistency of assumptions and model inputs in both ways (i.e. to IAMs as well as to other models) -- and participatory or collaborative modelling efforts where stakeholders from outside the respective modelling community are included. Examples for the former include the integration of IAMs with life cycle assessment techniques \parencite[e.g.][]{luderer_2019} and material flow analyses \parencite[for an overview see:][]{baars_2022,kullmann_2021}, and examples for the latter include projects where narratives and scenarios (unfortunately not the models themselves, though) are co-developed such as SHAPE. Moreover, one could hope to include also aspects that were previously ignored like power shifts \parencite{rutting_2022}. Such changes towards collaborative (including broader intercomparison exercises) and participatory procedures certainly face systemic barriers, for example regarding available funding and time resources, which have to be addressed.

On IAM scenario ensembles \url{https://www.nature.com/articles/s41558-022-01349-x}

For more trans-/interdisciplinary modelling, focusing on circular economy \url{https://doi.org/10.1016/j.spc.2021.12.011}

Concluding, socio-ecological systems are more flexible and thus exhibit more degrees of freedom than physico-chemical ones. Wicked problems occurring in such systems therefore cannot get addressed with reductionist approaches. Instead, more holistic ones are required. In terms of prospective analyses, this means that scenarios must be \textit{complete} and scenario sets must be \textit{exhaustive}. This requires sustainability scenario models implicitly to become more diverse and to abandon irrational and inconsistent assumptions - which permeates to the core structures of them. Overall, scenario-modelling should perhaps have fewer predictive and descriptive, but more exploratory (cf. breadth and depth of scenario space) and normative (cf. value judgements regarding scenario space) as well as collaborative elements (cf. how the scenario space is created). These must be clearly and transparently outlined so that flawed interpretations are prevented and that a \textit{fuller}, and thus potentially more robust exploration of the scenario possibility space is enabled on all levels and for all purposes. How the present proposal of \textit{real} scenarios relates to the characteristics of sustainability scenario analysis suggested by \textcite{swart_2004} is left to another exercise.

--
Suggestions:
- include "dummy extreme events", i.e. possible disruptions of some parameters along the way (at some time step)
- possibly create a "very worst case" scenario where all such possible disruptions happen (applied to all parameters)
- reduce sectoral (+spatial) resolution when further in the future

At the same time, we do not promote an "anything-goes" attitude. 


%It might be interesting to highlight the following point a little more, which is phrased for economic models but would similarly apply to IAMs etc: "that economic models are policy tools with both an overt objective function and a covert political function" and thus actively shape the political discourse and policies \url{https://www.tandfonline.com/doi/full/10.1080/13563467.2023.2172147?src}. While the intentions might be good, unreflected model set-up and use may have serious consequences.

%Some publications to check out:
%\begin{itemize}
%    \item \url{https://www.sciencedirect.com/science/article/pii/S2214629621003133?via%3Dihub}
%    \item \url{https://www.sciencedirect.com/science/article/pii/S001632872300099X?via%3Dihub}
%    \item \url{https://www.sciencedirect.com/science/article/pii/S0959652623017432?via%3Dihub}
%    \item \url{https://link.springer.com/article/10.1007/s43253-023-00098-7}
%    \item \url{https://wires.onlinelibrary.wiley.com/doi/10.1002/wcc.838} I would argue that these feasibility spaces are going in the wrong direction because feasibility, as a subset of plausibility, is entirely subjective. Opening perspectives through the lense of possibility might be more fruitful, especially when considering a plurality of modelling approached and schools of thought. Further, feasibility is constrained to only myopic actions, i.e. what is feasible right now, as everything more distant in the future might appear from the current standpoint as infeasible. Feasibility might be a good guiding light for necessary short-term changes but inadequate for large-scale transformation happening over a long period.
%    \item \url{https://arxiv.org/abs/2301.08135}
%    \item \url{https://www.nature.com/articles/nclimate2980}
%    \item \url{https://www.sciencedirect.com/science/article/pii/S1877343513000808?via%3Dihub}
%    \item \url{https://www.sciencedirect.com/science/article/pii/S2214629620304825?via%3Dihub}
%    \item \url{https://www.nature.com/articles/s41558-023-01681-w}
%    \item \url{https://www.sciencedirect.com/science/article/abs/pii/S2590332223000891}

%    \item \url{https://www.emerald.com/insight/content/doi/10.1108/14636680610668045/full/html}
%    \item \url{https://www.sciencedirect.com/science/article/pii/S0016328718303264}
%\end{itemize}
%

% On IO-IAM integration, this recent paper is a good summary \url{https://doi.org/10.1080/09535314.2023.2266559}

%add some notes on different types of scenarios, e.g. on comparative-static vs dynamic etc., which depends of course in part on the potential/ ability of the model to be used. This might be of interest because feasibility is much less of an important criterion (also because it cannot be really assessed/quantified) in the case of what-if scenarios for only a single point in time. Any model should however be able to reproduce past developments. Even for IO, this is not straightforward when it comes to impact analysis - imputation analysis is something very different. I wonder how that plays out for more complex models, especially IAMs.

%In terms of plausibility and possibility, I have to stress the role of statistical models used for scenario analysis (and their only limited suitability).

%The examples in the box of the WIRES paper are I think ill-suited because they do not take into account the aspect of time - WHEN! will something happen or someone do something?

%Desirability is another aspect to be considered, not necessarily linked directly to either of the terms possibility, plausibility, probability (and even feasibility).

%Research integrity also includes that researchers/modellers consider more than the carbon tunnel vision. That is, they must highlight that their analysis focuses mainly on climate change and that trade-offs of environmental impact are possible and that climate change mitigation and adaptation may come at the cost of exacerbating other environmental and social impacts.

%The definition of feasibility at one point in time hinges on what has happened in the years before. There is therefore some contingency - the definition is time-dependent

%Especially when complex dynamics are involved (such as trade, any kind of social dynamics), "simple" forecasting or predictions are not appropriate anymore - at least when trying to capture big pictures. For very narrow (type and spatiotemporal) cases, that might be still justified though.

% This review on scenario has a short section on scenario adaptation and customisation. It also distinguishes between risk and exploratory scenarios, and addresses the problem of putting numbers on qualitative factors \url{https://www.annualreviews.org/doi/full/10.1146/annurev-environ-102017-030109}

\newrefcontext[sorting=nyt] % sort the paper by name, year, title
\printbibliography[heading = bibintoc] % 'bibintoc' inserts our bibliography into the table of contents

\end{refsection}

%%%%%%%%%%%%%%%
%% Appendix
%% Inserting appendix with separate settings
%\newpage
%\setcounter{page}{1}
%\renewcommand{\thepage}{A-\arabic{page}}
%\linenumbers*
%\addappendix
%
%%Reset numbering of tables and equations in appendix, starting with A.
%\renewcommand{\thetable}{A.\arabic{table}}
%\setcounter{table}{0}
%\renewcommand{\theequation}{A.\arabic{equation}}
%\setcounter{equation}{0}
%
%\begin{refsection}
%\section*{Appendix A}
%BLABLA
%
%\section*{Appendix B}
%blabla \parencite{leontief_1936}
%
%\nolinenumbers
%\newpage
%\newrefcontext[sorting=nyt] % sort the paper by name, year, title
%\printbibliography[title = References in appendix]
%
%\end{refsection}
\end{document}