\section{An introduction to...}

... economics and ecology ought to point out that the two surely have more in common than their etymological origin. And yet, while humanity's immediate dependence on its natural surroundings lies in plain sight, economics gradually reduced in its own considerations the latter to an abstract banality. From these segmented externalities regarded by Pigouvian taxes \parencites{pigou_2017}[see also][]{baumol_1972}, Coase's \parencite*{coase_1960} social costs, and the like -- and despite earlier fundamental insights by \textcite{carlowitz_1713}, \textcite{jevons_1865}, \textcite{arrhenius_1896}, and their respective contemporaries -- it was a long way until socio-ecological systems were again contemplated holistically, as was the case in the day of Aristotle. The seed for this imminent transition came not only from within economics, nor from ecology, but also included fields like environmental literature and philosophy; one may think of the works by Waldo Emerson, Henry David Thoreau, John Muir, Aldo Leopold, and Arne Næss.

Starting with generic musings by \textcite{daly_1968,isard_1968,cumberland_1966}, \textcite{ayres_kneese_1969} eventually presented in a seminal study how nature-society interactions could be acknowledged more systematically. One can only speculate about their motives of doing so in a neo-classical Walras-Cassel general equilibrium model and not in one more agnostic to economic schools of thought. Yet, before long, interest in this kind of academic exercise spiked rapidly. This was driven by crucial observations like those of Rachel \textcite{carson_1962}, new conjectures on the presumable roots of ecological crises \parencite{white_1967}, and powerful ideas such as the Gaia hypothesis \parencite{lovelock_1972,lovelock_1974}, the concept of a spaceship Earth \parencite{boulding_1966}, and the $I=PAT$ equation \parencite{ehrlich_1972,commoner_1972,chertow_2001}.

Parallel to important contributions in economics on the attribution of current environmental repercussions \parencite[see for example][]{leontief_1970,leontief_1972}, controversial neo-Malthusian concerns à la \textit{Tragedy of the Commons} \parencite{hardin_1968,ostrom_1999,mildenberger_2019} paved the way for the emerging field of world modelling, pioneered by the early activities of the Club of Rome. Different from siloed scientific approaches before, this new field had at its focus the holistic analysis of complex systems such as those which human-nature interactions comprise. Although coarse in detail, initial models like World3 used in \textit{The Limits to Growth} \parencite{meadows_1972} were groundbreaking. Soon after, more refined ones were developed to address the \textit{World Problematique}. The United Nations were instrumental in furthering this work \parencite{fontela_2004}, where some models emphasised the economic aspects (in the tradition of meso-economic and general equilibrium models, e.g. \textcite{leontief_1974}) whereas others focused on interactions within a system of systems (as in systems dynamics, e.g. \textcite{mesarovic_1974}).

With publications such as \textit{Only One Earth} \parencite{ward_1972} and \textit{Our Common Future} \parencite*[WCED,][]{brundtland_1987} documenting the rising necessity to reconsider then-current socio-economic trajectories, the need for quantitative models kept on increasing. Already back then, the important idea of inner and outer limits under various spatio-temporal considerations arose which is popularised today under the name of an inter- and intragenerationally equitable doughnut economy where human needs are satisfied within planetary boundaries; a concept that is in some aspects also approximated by the ideas of deep ecology and strong sustainability. Thus, it is not surprising that the scope of many computational models expanded, both in breadth and in detail. In the following years, various interdisciplinary models (often under the umbrella term of energy-economy models) emerged and first steps in the direction of truly integrated models were undertaken. The meaning of integration goes here in two ways: vertical integration in the sense of linking causes and consequences of environmental issues along their causal chain, and horizontal integration denoting the breadth of causal links as well as cross-topic representations \parencite{parson_1997}. This level of integration has certainly stood in contrast to the granularity of the system representation, and continues to do so today \parencite{krey_2014}.

Although the relevance of a changing climate was acknowledged in the scientific community long before James Hansen's testimony to the US Congress, the first proper integrated assessment model (IAM), RAINS \parencite[developed during the 1980s; for an overview see][]{alcamo_1991}, dealt with acid rain. Yet, RAINS served as a blueprint for similar models that would address climate change and other topics. Such models became relatively quickly part of decision-making processes; as was the case with, for example, ASF \parencite{lashof_1989} and IMAGE \parencite{rotmans_1990} in the first report by the Intergovernmental Panel on Climate Change \parencite*[IPCC,][]{ipcc_1991}.

When speaking of IAMs, one must distinguish between cost-benefit and process-based models: the former are in direct lineage of Coase's concept of social cost and most often are highly aggregated models, whereas most process-based models emerged from the literature on energy modelling and only few were elevated from meso-economic models.\footnotemark{}

\footnotetext{Cost-benefit models such as PAGE \parencite{yumashev_2019}, FUND \parencite{waldhoff_2014}, and DICE/RICE \parencite{nordhaus_2017} are also termed policy optimisation models, whereas process-based models such as IMAGE \parencite{stehfest_2014}, MESSAGE \parencite{huppmann_2019}, and REMIND \parencite{baumstark_2021} are often referred to as policy evaluation models. The solution concept -- optimisation or simulation -- is, however, not the eponym for these two classes of models; it is rather their use that is the reason for their naming. Cost-benefit models are typically employed to identify the \textit{optimal} balance (based on some criterion) between costs and benefits of policies and impacts. In contrast, process-based models are used to evaluate socio-economic and environmental consequences of actions and impacts; depending on their level of integration and system representation, process-based models are sometimes also referred to as energy-economy-environment (E3) or E4 (E3 + engineering) models. Some models, such as WITCH \parencite{bosetti_2006,emmerling_2016}, can be employed for both policy optimisation and evaluation. In fact, cost-benefit and process-based IAMs may be calibrated to the results of the other \parencite{vanden_2020}, as well as against more complex impact models \parencite{moss_2010}. Both types of IAMs have repeatedly been subject to fundamental criticism \parencite[see for example][]{stern_2016,weyant_2017,keen_2021,gambhir_2019,farmer_2015,pindyck_2017}.}

The spawn of IAMs coincided with sustainability concerns seeping into other scientific disciplines. Earth system models, for example, expanded from modelling only biogeochemical processes of the Earth system to also account for human drivers of changes \parencite{flato_2011}.\footnotemark{} And while it seemed for a while that sociology was slow to pick up on it \parencite{passerini_1998}, it is now rather the case that its potent contribution to sustainability discourses is overlooked \parencite{longo_2021}. Meanwhile, in one branch of engineering, nature-society interactions started to be examined on a rather granular level while still maintaining a system's perspective \parencite{frosch_1989,ausubel_1992}. Characterised by its ideal of cyclic material flows and holistic views \parencite{jelinski_1992}, this new field of industrial ecology thus anticipated the concept of a circular economy \parencite{kalmykova_2018,merli_2018} and, together with other approaches firmly rooted in the respect for thermodynamic principles \parencite{haberl_2019}, the scientific paradigm of socio-economic metabolism \parencite{pauliuk_2015}. An emphasis on thermodynamic principles and roots in physiocratic thinking were also crucial for the emergence of ecological economics, a branch of economics that started growing after, and was moulded by the likes of Robert Ayres, Kenneth Boulding, and Nicholas Georgescu-Roegen \parencite{cleveland_1999}.

\footnotetext{The literature knows many attempts of coupling these models with IAMs. Although primarily \textit{simple} climate and Earth system models such as MAGICC \parencite{meinshauen_2011,meinshausen_2020} have actually been integrated into IAMs \parencite{vanVuuren_2011}, couplings of more complex models have been conducted as well and are expected to become more common \parencite{vanVuuren_2012}.}

Thus, one may observe today a large heterogeneity of models and model families used to analyse socio-ecological systems. These differ not only in their origins and traditions, but also in terms of the aspects they focus on and their underlying principles. For example, IAMs often do not explicitly represent society's biophysical basis, while this is essential to industrial ecology approaches \parencite{pauliuk_2017}. And while industrial ecology and ecological economics share assumptions and considerations, their detail of analysis often differs \parencite{kronenberg_2006}. In addition, one can identify even within model families large differences, such as regarding solution concepts in IAMs or the scale of analysis in industrial ecology.

While some of these differences are not unproblematic, for instance regarding the (dis)respect for thermodynamic laws, one could argue that despite their individual flaws such a large variety of approaches is required to adequately understand socio-ecological systems. Even more so given that in the short time span of the anthropocene \parencite{crutzen_2006}, we have moved quickly from only feeding on nature's free gift \parencite{karsten_1987} to grossly altering our surroundings on all levels in a full world \parencite{daly_2005}. Reports by the IPCC \parencite[e.g.][]{ipcc_2022}, the intergovernmental science-policy platform on biodiversity and ecosystem services \parencite*[e.g. IPBES,][]{ipbes_2019}, and the international resource panel \parencite[e.g.][]{unep_2016} have repeatedly documented humanity's voracity and its consequences. And yet, despite substantially transgressing Earth's biophysical boundaries and getting close to tipping points of all sorts \parencite{ceballos_2015,steffen_2015, steffen_2018}, we are failing on a large scale to achieve social targets \parencite{fanning_2022}.

It is therefore imperative for sustainability science and the associated disciplines to not only understand socio-ecological systems in hindsight \parencite{kates_2001} but also to provide prospective guidance by outlining potential futures \parencite{swart_2004}. This is underlined by both the precautionary principle\footnotemark{} as well as humanity's ancient wish to foresee the future. Although predictions of such complex systems as the human and natural systems are practically impossible, other ways of addressing the future have been devised \parencite{polasky_2011}, with scenario-modelling being well-established in sustainability science. While scenario modelling is the ultimate purpose of IAMs, it has rather slowly been picked up in other domains of sustainability assessment. The IAM community has therefore, and because of their models' comprehensive integration, an advantage as compared to other approaches. Scenario sets originating from the IAM community are highly authoritative and form the basis for global decision-making with regard to climate and other environmental change. Such scenario sets include in particular the representative concentration pathways \parencite[RCPs;][]{vanvuuren_2011_rcp} and the shared socio-economic pathways \parencite[SSPs;][]{riahi_2017,oneill_2014} in the context of climate change. Comprehensive scenario sets for the wider sustainability perspective are less common \parencite{vansoest_2019}, but include notably the sustainable development pathways developed by The World in 2050 initiative \parencite{twi_2018} and those by the SHAPE project \parencite[e.g.][]{soergel_2021}. In addition, modelling approaches from other domains of sustainability science often adopt the assumptions, input parameters, or scenario outputs from these authoritative sources. Examples include the studies by \textcite{deetman_2018,beltran_2020,schandl_2020}. Alternatively, modellers from other fields might devise their own scenarios or rely only partially on larger scenario sets \parencite[for example][]{pauliuk_2013}.

\footnotetext{It should be noted, though, that there are of course different (value-laden) perceptions of the precautionary principle \parencite{gardiner_2006}.}

In what follows (section \ref{main_part}), I contend that current sustainability scenario-modelling efforts are insufficient in that they are: i) dominated by one model family from which its shortcomings descend to other approaches (or alternatively, if sector-specific scenarios are crafted with other model families, that those might be inconsistent); ii) highly sensitive to model assumptions, structure, and parameterisation; and iii), because of this incomplete exploration of the possibility space, biasing the wider scenario discourse and further scenario modelling efforts. To resolve these issues, I suggest to move beyond conventional efforts and over to \textit{real} scenario-modelling. With that, I mean that each scenario set should be: a) \textit{complete}, and b) \textit{exhaustive}. In addition, c) I argue that both scenario sets and respective models should enhance \textit{deep} integration. What these terms imply is discussed in section \ref{discussion}.





