\section{An unfortunate situation}

Scenarios are generally characterised as qualitative or quantitative manifestations of plausible and internally consistent narratives about the future. The antecedents of today's scenario-modelling exercises lie in war games, from where the use of scenarios spread to strategic planning in business and other organisations \parencite{bradfield_2005,schoemaker_1993}.\footnotemark{} With environmental concerns growing during the 1960s-70s, and emphasised by institutions like the Club of Rome and the United Nations, long-term scenario-modelling started to be also used for illuminating potential futures of socio-ecological systems.

\footnotetext{Herman Kahn in the US and Gaston Berger in France were presumably the two most influential figures in these developments.}

Great strides have been made since these pioneering endeavours. And it seems that still more has to be done: While some of the problems addressed by earlier efforts appeared to be only one-dimensional and thus could be dealt with comparatively straightforwardly, we are currently experiencing polycrises in our increasingly volatile, uncertain, complex, and ambiguous world. Some of these crises (individually or combined), may be considered as wicked \parencite{rittel_1973, termeer_2019,lonngren_2021} or even super-wicked problems, such as climate change \parencite{levin_2012}. These are vastly more complex and difficult to solve than other problems that seemed insurmountable at first, such as the ostensible 1894 great horse-manure crisis or the issue of acid rain in late-20$^{th}$ century Europe. These current problems demand therefore more sophisticated approaches, both regarding problem understanding (i.e. modelling) and solving (i.e. policy responses). 

Today's sustainability scenario-modelling is largely dominated by the IAM community, their models, and the scenarios that they develop. While the development process for the most authoritative global change scenarios has evolved from moving only sequentially along the causal chain to a more parsimonious, parallel process \parencite{moss_2010} and ultimately combining scenario sets in a matrix architecture \parencite{vanvuuren_2014},\footnotemark{} SSPs rely just like RCPs (and often other frameworks, too) on marker scenarios. That is, for each of a set of agreed upon narratives there is one representative scenario implementation chosen. Non-marker scenarios can then be understood as alternative scenario interpretations \parencite{riahi_2017}. The marker scenarios are, however, the authoritative ones and are exclusively implementations made with major IAMs such as AIM \parencite{fujimori_2017}, GCAM \parencite{calvin_2017}, and MESSAGE \parencite{fricko_20171}. When such scenarios are then taken over by other assessment methods, it is usually only the marker scenarios that are employed, thus introducing a bias towards the bigger IAMs (and thus towards large modelling teams that might also receive comparatively more funding). Also across different scenario sets, there is a bias towards larger IAMs: among all 1,686 vetted pathways explored in the IPCC Sixth Assessment Report database using a pool of more than 20 models, about one third was computed using MESSAGE and REMIND \parencite{gambhir_2022,byers_2022}.

\footnotetext{Although many goals were achieved with this scenario architecture, the common sentiment among respective modellers is that further improvements must be made, such as the integration of societal conditions \parencite{oneill_2020}. Others have raised their voices concerning the plausibility of some scenario combinations, such as SSP3-RCP7.0 or RCP5-SSP8.5 \parencite{pielke_2022}.}

On the other hand, these major IAMs have some of the most comprehensive socio-ecological system representations available. Other scenario-modelling approaches that focus only on a subsystem (e.g. a sector, or in terms of a partial equilibrium) might yield inconsistent modelling outcomes and scenario sets. The reason for this is simply that feedback effects (including rebound effects) might not be accounted for, thus potentially resulting in inconsistent dynamics of the system. In addition, insights from such exercises may be hard to compare to the outcomes from more comprehensive (although perhaps coarser) assessments.

 Among IAMs, there are great differences in terms of model structures, input assumptions, and model outcomes \parencite{krey_2014,krey_2019,hiroto_2020,keppo_2021}. It may thus happen that one ends up comparing apples and oranges. For example, when a model (say, one relying on intertemporal optimisation) is fundamentally different from another one (e.g. a myopic systems dynamics model), and when their data parameterisations of the same aspects differ, too, one cannot expect to obtain like interpretations for a common exercise.\footnotemark{} Although the issue has already been recognised \parencite{giarola_2021}, there may continue to be cases where modelling outcomes are more sensitive to the choice of IAM than to the input parameterisation \parencite{sognnaes_2021}. At the same time, the case of similar models yielding different results because of only different parameterisations is hardly surprising either. On an even more fundamental level, IAMs might diverge in their worldviews, with many among the cost-optimizing frameworks following neoclassical assumptions. Well-known problematic aspects following from these worldviews often include the application of discount rates \parencite{emmerling_2019}, non-coverage  of capital trade (because of the otherwise occurring Lucas paradox \parencite{lucas_1990,keppo_2021}), a disregard for emerging behaviours \parencite{farmer_2015}, or unrealistic behavioural assumptions \parencite{asefi_2021}. Another issue is certainly when input or output parameters from one model (which might be based on neoclassical assumptions) are uncritically adopted by other modelling efforts that might have entirely different (and thus inconsistent) assumptions (e.g. post-Keynesian) - or where even the parameter sets are contradictory for some variables. It is even more problematic when results from these secondary efforts might find their way back again into the first set of models, thus creating a wild mix in terms of assumptions and parameterisations.\footnotemark{}

\footnotetext{On that note, it need be mentioned that the selective use of some scenarios and their (mis)interpretation is a topic in itself which has been addressed multiple times already \parencite{pielke_2021}, e.g. regarding the use of RCP8.5 as a baseline scenario \parencite{hausfather_2020}.}

\footnotetext{Importantly, some of the input assumptions of IAMs may be too coarse to use as an input for more disaggregated secondary modelling. It should be noted, however, that there also exist multi-sectoral IAMs of higher resolution; they do rely on different assumptions and principles, though \parencite{lefevre_2022}.}

These and other issues are the reasons for why it has been argued that IAMs do not explore the full possibility space \parencite{keppo_2021,mccollum_2020, gambhir_2022}. Assumptions integral to individual models then become inevitably implicit in scenario sets developed with these models. Since explicit and implicit value choices are made in the construction of both the IAMs and their scenario sets (and the underlying narratives, respectively), care must be applied when interpreting the outcomes of such efforts \parencite[even if it is only reference scenarios; see e.g.][]{grant_2020} - and even more so when adopting them for further (secondary) analysis with models from other domains. Given the relevance of global change scenarios for policy-making, one moves thus quickly from questions of intrinsic to extrinsic ethics, meaning that the wider society might be affected by modellers' judgements \parencite{beck_2016},\footnotemark{} for instance in terms of implicit justice perceptions \parencite{rubiano_2022}. One might even go so far as to allege a bias towards Western value sets, given that most of the IAMs are being developed in respective countries which are then also predominantly employed in multi-model comparison studies \parencite{duan_2019}.

\footnotetext{Given that wicked problems such as those addressed with IAMs cannot be "tamed" with only a reductionist approach \parencite[nor is it morally justifiable to do so;][]{churchman_1967}, especially not under the precautionary principle, ethical values come inevitably into play when widening the analysis' subject and the respective comprehensive modelling. This is typically the case when social dynamics are part of the problem -- as is common for wicked problems.}

As of now, IAMs are our best shot to tackle large-scale (or even wicked) problems comprehensively, but they fail on different levels. Especially the incomplete (and sometimes inadequate; cf. neoclassical assumptions) coverage of socio-economic dimensions in these models creates a bias towards irrational (although some economists might call it rational for exactly that) futures. Such model assumptions and biases are perpetuated and disseminated through respective scenarios to secondary modelling efforts and to the wider discourse on topics of global change. It is therefore that I argue that we need to move beyond these conventional scenario-modelling efforts and towards \textit{real} ones.