\section{...and a plea to better it}

With \textit{real} I do not mean realistic scenarios (nor even plausible ones, for that matter). Rather, we have seen great transformations in human history before \parencite{fischer_2014} so that I believe that scenario-modelling exercises should account for possible (again, not necessarily plausible from the current viewpoint), disruptive developments in the future.\footnotemark{} Support for this view comes also from within the IAM community. For example, \textcite{mccollum_2020} argue that IAMs should more vehemently explore extreme scenarios so as to account for transient (i.e. low-probability) events (e.g. sudden and widespread recessions), disruptive drivers (e.g. disintegration of political alliances), and unexpected outcomes (e.g. technological change). \textcite{gambhir_2022} point out that current IAMs do not allow for such a more complete exhaustion of the possibility space. Of course, a full exploration of the scenario space is virtually impossible, but I would nevertheless urge for a \textit{fuller} exploration than what is currently done. This relies on three points that I shall briefly explain in the following.

\footnotetext{Even if one disregards the great transitions of the past, one only has to consider the fact that current scenario models of global change explore the future to the year 2100 - if we just look back at what the world was like 80 years or so ago, still in the middle of WWII, one might realise that drastic changes have happened that were hard to foresee back then. Hence, more or less linear extrapolations of current conditions and thinking are grossly inappropriate, thus essentially ruling out all approaches that envision a future strongly based on the past. This is the case for practically all approaches based on statistical learning methods and for narratives that stick too closely to past experiences.}

First, each individual scenario must be \textit{complete}. With that I mean that a scenario must fully and transparently represent the entire system, not only a subsystem. This does not mean that every model employed necessarily has to be capable of a quantitative full system coverage; rather, the model must respect fundamental principles such as thermodynamic laws and produce a scenario that is in so far consistent and coherent as that feedback effects are sufficiently accounted for and that no contradictions are prevalent, both regarding its implementation and the narrative that it is based on.

Secondly, each scenario set must be \textit{exhaustive}. That is, a sufficiently large set of scenarios must be constructed which includes not only those that are thinkable given the current conditions but also those that are considered as (too) extreme. This relates to both more exploratory parameterisations of common narratives as well as the compilation of more out-of-the-ordinary narratives. A relatively mild version of the latter includes current post-growth considerations \parencite{hickel_2021}. An \textit{exhaustive} scenario set certainly requires also to have a wider range of models producing the individual (\textit{complete}!) scenarios (and thus perhaps to give up the concept of marker scenarios), such that more alternative worldviews and model philosophies are covered and find their way more easily into secondary modelling and policy discourses.\footnotemark{} We have seen in the past that transient events caused by great disruptions under fundamental uncertainty have happened. In future scenarios we have to account for them, too, via all means possible. If that is done under assignment of probabilities or not is secondary.

\footnotetext{This means to also expand from the over-reliance on current modelling frameworks, like the cost-optimisation ones (e.g. MESSAGE) or the ones based on systems dynamics (e.g. IMAGE), to a more open landscape where alternative approaches based on e.g. agent-based modelling are sufficiently well represented, too. I do not argue, though, that the current major IAM modelling groups are solely to be blamed for the current situation; it is rather an unfortunate course of events (due to selective scenario applications and misinterpretations of individual scenarios and their implicit and explicit assumptions) that has put too much emphasis on only a limited set of models with limited worldviews and the thus produced scenarios.}

In addition, I argue that sustainability scenario assessments have to not only increase their vertical and horizontal integration (while not downgrading the level of representation to inappropriate levels), but also to enhance \textit{deep} integration. With that I mean both the integration of IAMs with other modelling approaches -- in the sense of a multi-model ecology \parencite{bollinger_2015} so as to improve the consistency of assumptions and model inputs in both ways (i.e. to IAMs as well as to other models) -- and participatory or collaborative modelling efforts where stakeholders from outside the respective modelling community are included. Examples for the former include the integration of IAMs with life cycle assessment techniques \parencite[e.g.][]{luderer_2019} and material flow analyses \parencite[for an overview see:][]{baars_2022,kullmann_2021}, and examples for the latter include projects where narratives and scenarios (unfortunately not the models themselves, though) are co-developed such as SHAPE. Moreover, one could hope to include also aspects that were previously ignored like power shifts \parencite{rutting_2022}. Such changes towards collaborative (including broader intercomparison exercises) and participatory procedures certainly face systemic barriers, for example regarding available funding and time resources, which have to be addressed.

Concluding, socio-ecological systems are more flexible and thus exhibit more degrees of freedom than physico-chemical ones. Wicked problems occurring in such systems therefore cannot get addressed with reductionist approaches. Instead, more holistic ones are required. In terms of prospective analyses, this means that scenarios must be \textit{complete} and scenario sets must be \textit{exhaustive}. This requires sustainability scenario models implicitly to become more diverse and to abandon irrational and inconsistent assumptions - which permeates to the core structures of them. Overall, scenario-modelling should perhaps have fewer predictive and descriptive, but more exploratory (cf. breadth and depth of scenario space) and normative (cf. value judgements regarding scenario space) as well as collaborative elements (cf. how the scenario space is created). These must be clearly and transparently outlined so that flawed interpretations are prevented and that a \textit{fuller}, and thus potentially more robust exploration of the scenario possibility space is enabled on all levels and for all purposes. How the present proposal of \textit{real} scenarios relates to the characteristics of sustainability scenario analysis suggested by \textcite{swart_2004} is left to another exercise.




It might be interesting to highlight the following point a little more, which is phrased for economic models but would similarly apply to IAMs etc: "that economic models are policy tools with both an overt objective function and a covert political function" and thus actively shape the political discourse and policies \url{https://www.tandfonline.com/doi/full/10.1080/13563467.2023.2172147?src}. While the intentions might be good, unreflected model set-up and use may have serious consequences.