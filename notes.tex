\documentclass{article}
\usepackage{packages}
\begin{document}

Touch upon WHY some models employ parameterised scenarios generated with other models... basically, because they are (or cannot) be (easily) integrated with other models, e.g. a climate model into an LCA or MFA system

Coupled models, model intercomparison, and sustainability scenario modelling in non-IAM fields might run into the same problems as IPCC scenarios have already (cf. Pielke \url{https://doi.org/10.1016/j.erss.2020.101890} and \url{https://iopscience.iop.org/article/10.1088/1748-9326/ac4ebf}), with a special case when secondary modelling relies on such IPCC scenarios

What are scenarios? In the context of sustainability
science, integrated scenarios may be thought of as
coherent and plausible stories, told in words and
numbers, about the possible co-evolutionary pathways
of combined human and environmental systems. (Swart)
It is becoming increasingly clear that such pathways need to recognise the dependence of the human on the environmental systems. (Pauliuk Hertwich)
Adopt the view of van Vuuren that statements on probability are avoided when talking about scenarios, thus distinguishing scenarios from forecasts \url{https://www.sciencedirect.com/science/article/pii/S0959378012000635}, chapter 2.2.1. Importantly, also make the distinction between exploratory and normative (=policy) scenarios. Right now, I think I advocate mainly for broader exploratory scenarios, but what does that mean for the use of normative scenarios; also, what does it mean when primary models put together exploratory/normative scenarios which are then used in normative/exploratory scenarios by secondary models? (This question is based on my assumption that many IE scenario assessments are rather normative, especially prospective LCAs? although it's tricky because the latter very often outfit their scenarios very selectively). Van Vuuren link the two different types to stages in the policy-making process (problem recognition, policy formulation, policy implementation, control), e.g. exploratory scenarios for getting issue on table, but also offer other explanations for the seemingly increased shift from exploratory to policy scenarios (i.e. earlier critique of exploratory scenarios as well as differences across scientific disciplines). The authors remark that a shift to polciy scenarios bears the risk of being too worldview dependent and ignoring considerable risks/uncertainties.
As many models are calibrated on past and current conditions, they might only be able to address "shallow leverage points", as suggested for land use models by Verburg et al. \url{https://www.sciencedirect.com/science/article/pii/S1877343518301362#bib0355}, picking up the "leverage points perspective on sustainability" by Fischer and Riechers \url{https://besjournals.onlinelibrary.wiley.com/doi/full/10.1002/pan3.13}; I suppose this is mainly true for models extrapolating previous trends (e.g. statistical models) but might be less so for others, e.g. optimisation-based IAMs? As for the leverage points perspective (how does it relate to regime shifts and transitions btw?), I suppose that this goes in a similar direction as my point on including more socio-dynamic elements in modelling (as they may have bigger impacts wrt sustainability than only efficiency improvements and the like)

Swart paper as guiding element, and then framed via Elsawah's grand challenges (with a new one emerging)

Not sure if the review by Clark and Harley ("Sustainability Science: Toward a Synthesis") is relevant and should be cited \url{https://www.annualreviews.org/doi/full/10.1146/annurev-environ-012420-043621}

A little more useful might be the review "Exploring Alternative Futures in the Anthropocene" by Cork et al. \url{https://www.annualreviews.org/doi/full/10.1146/annurev-environ-112321-095011}

IAM scenario work often (mainly?) follows the story and simulation approach.
The "big" IAMs are IAMs for climate change. Their resolution of activities without significant influence on the climate is low, thus making them ill-suited to address other impact categories.

Remark somewhere that sustainability scenario modelling is important as it can inform the "what" and, possibly, the "how" of transformation. On the what and how see \url{https://doi.org/10.1007/s11625-022-01123-0} and on the difference between transition and transformation \url{https://www.sciencedirect.com/science/article/pii/S2210422417300801}

Could it be a problem when short-term of scenarios is qualified by plausibility and long-term by possibility? That is, when significantly opening the solution (and target?) space for the distant future, would some models (e.g. optimisation-based) distort metrics/indicators, e.g. an increased use of so-far unexistant NETs in distant future to end up with ok climate change?

Maybe useful to add that there's (seemingly) no agreement on the definition of the term "scenario" \url{https://onlinelibrary.wiley.com/doi/pdfdirect/10.1002/ffo2.3}. In the same paper it is also mentioned that scenario has its etymological origin in the Latin word for stage, and that there is a difference between scenario planning and scenario development (the latter being, I think, applicable in our context, although scenario planning could be relevant if analysis on small scale, e.g. very detailed LCAs that are moreover informed by organisational decision-making?)

As for a classification of scenario, I might just fall back on the typology by Börjeson, which is also described in Bishop et al \url{https://www.researchgate.net/profile/Andy-Hines/publication/228623754_The_current_state_of_scenario_development_An_overview_of_techniques/links/0f31753c036cf1a857000000/The-current-state-of-scenario-development-An-overview-of-techniques.pdf}

Other typologies and classifications are provided/summarised by Amer et al. \url{https://www.sciencedirect.com/science/article/pii/S0016328712001978}, or earlier by van Notten et al. \url{https://www.sciencedirect.com/science/article/pii/S0016328702000903}

Van Vuuren et al. describe some useful scenario terminology, e.g. baseline scenarios, BAU scenarios, back-casting \url{https://www.sciencedirect.com/science/article/pii/S0959378012000635}. Pick this up later when it comes to extreme events as they might not be captured by any meaningful probability distributions, i.e. black swans etc..

This article by Elsawah et al. seems to be very comprehensive when it comes to the scenario generation process \url{https://www.sciencedirect.com/science/article/pii/S0048969720319069}

This paper by Elsawah et al. identifies eight grand challenges in socio-environmental systems modelling \url{https://sesmo.org/article/view/16226/15789}. Maybe useful in combination with Swart paper to narrow down our contribution? (As well as the review paper by Wiebe et al. that also mention difficulties arising from scenario adaptation and customisation \url{https://www.annualreviews.org/doi/full/10.1146/annurev-environ-102017-030109}). Maybe one could argue that the feedback loop of assumptions into assumptions presens a separate grand challenge (admittedly a not obvious one which is ssimultaneously just emerging now when more and more (sustainability) modelling communities jump onto the train of prospective modelling), lying right at the interdisciplinary section of primary scenario developers and users/customisers?

This overview article by Guivarch mentions that "too ittle attention is given to the implication of plausible alternative model functional forms", thus asking a question about model comparability and the definition of plausibility embedded there \url{https://www.sciencedirect.com/science/article/pii/S1364815217303936}. The article is mainly however an overview of recent additions to scenario techniques.

Maier et al. on multiple plausible futures and its relation to deep uncertainty \url{https://www.sciencedirect.com/science/article/pii/S1364815216300780}

Possibly a good review on scenario techniques and the topic of black swans; they also develop a new technique \url{https://www.sciencedirect.com/science/article/pii/S0016328722000398}

And this article appears to classify existing typologies \url{https://www.sciencedirect.com/science/article/pii/S0040162519306924}

An overview of what makes socio-environmental modelling more useful to policymakers \url{https://besjournals.onlinelibrary.wiley.com/doi/pdfdirect/10.1002/pan3.10207}

I think that the essence of my time-dependent plausibility critique might be captured, at least partially, by "Moreover, the scenario users are reflexive in their activities: prepared not only to change their route and destination, but also to remodel their ship and its crew." as Wilkinson write \url{https://iopscience.iop.org/article/10.1088/1748-9326/3/4/045017/pdf}

Selin et al. provide a good overview of recent progress in dynamic sustainability scenario modelling, writing that many old challenges are started to get addressed \url{https://www.pnas.org/doi/abs/10.1073/pnas.2216656120}

Donges et al. distinguish between different taxa (=subsystems) of world-Earth systems. I suppose one might argue that scenario models should be designed in such a way that the do not violate any fundamental law in any taxon or at any interaction between taxa. \url{https://esd.copernicus.org/articles/12/1115/2021/} At the same time, some laws might of course still be unknown (known and unknown unknowns), such that models/assumptions might violate them unknowingly. Also, model abstraction might inadvertently violate such laws (when some dynamics are not represented sufficiently).

Aside from climate mitigation scenarios, there's now a big push from (IAM-)modellers to also account for nature more comprehensively: the nature futures framework "emerged to inspire the development of nature and people positive, diverse values-integrated, and multiscale scenarios" \url{https://www.sciencedirect.com/science/article/pii/S095937802300047X} and \url{https://doi.org/10.1016/j.gloenvcha.2023.102681}. That framework overlaps well with Donges' world-Earth system taxonomy and earlier concept of techno- and ecosphere or also Stefan's paper on socio-economic metabolism.

Similarly, there's another global set of scenarios on land use and food security, including one with a "perfect storm" of climate change and ecological crisis (haven't read details though) \url{https://journals.plos.org/plosone/article?id=10.1371/journal.pone.0235597}. The authors mention other global scenario sets that have different foci: climate change, ecosystem changes, land use + agriculture + food... They also cite papers that question the 2x2 scenario method and therefore advocate morphological analysis (I believe this is similar to how SHAPE did it)

One thing that should also be clear (and highlighted) is that there might be conflicting (scenario-)targets! Hence, the discussion on possibility space and plausibility might have to be widened, simply so that more comprehensive explorations of multiple objectives can be examined. (the nature futures scenarios might be a good example for that when put next to most of the climate mitigation scenario literature). I should add some studies highlighting for instance the conflicting nature of the SDGs (going basically back to wicked problems). Moreover, some very popular scenario sets that are sometimes used for "general" sustainability modelling have actually specific foci, such as the RCPs/SSPs on climate mitigation and adaptation. Also this better literacy of scenario applicability is necessary to improve.

Perhaps distinguish in secondary models between those doing basically just accounting and those that capture dynamics (i.e. essentially the attributional-consequential divide). The call for better model/scenario/assumption understanding goes to both, although at differing degrees of "severity". Also, the accounting models might rely on sometimes strong assumptions, depending on how abstract they are. Some approaches can be very problematic (at least conceptually), such as prospective attributional LCA where data from multiple time series is taken to analyse an impact for a particular point in time (e.g. taking time-ignorant ecoinvent data and combining it with time-step specific IAM data as well as "dependent" (i.e. dynamic) foreground data.)

Introduce socio-economic metabolism and SEM research (refer among others to this Nature review \url{https://www.nature.com/articles/s41893-019-0225-2})

Schweizer and Kriegler on internal consistency of scenarios, exemplified via their assessment (via cross-impact balance) and remodelling of SRES scenarios \url{https://iopscience.iop.org/article/10.1088/1748-9326/7/4/044011/meta}

The primary vs secondary framing might still make sense as it seems that many of the prospective LCAs have used non-IAM data in their background. And there are those IAM studies that use LCA and/or MFA data...

Arguing for more extreme event exploration might be supported by social tipping points not being adequately reflected in existing modelling efforts - but which might have high leverage. Engels and Marotzke have perhaps some good references \url{https://iopscience.iop.org/article/10.1088/1748-9326/acaf90/pdf}

Important to distinguish among prospective models those that simulate (primarily IAMs, but also MFAs and some IO/LCA-LPs) and those that account (MFAccounting, IO, LCA etc.)

Representing IAMs as either primary or secondary models is of course problematic as IAMs are already a collection of models. In what follows, we will nevertheless regard a single IAM framework as a single model, either as primary or secondary.

This paper argues for adoption of scenarios in physical climate modelling, explaining the reasons through an example - perhaps something similar could be good for our paper, i.e. describe a hypothetical situation where primary and secondary models interact and where the scenario is then qualified?... \url{https://link.springer.com/article/10.1007/s10584-018-2317-9}. Similar to how they conclude, start my introduction with reference to the post-normal science paper, thus justifying the use of scenarios; this is particularly important for intricate problems such as those posed by sustainable development... \url{https://doi.org/10.1016/0016-3287(93)90022-L}

It might be interesting to highlight the following point a little more, which is phrased for economic models but would similarly apply to IAMs etc: "that economic models are policy tools with both an overt objective function and a covert political function" and thus actively shape the political discourse and policies \url{https://www.tandfonline.com/doi/full/10.1080/13563467.2023.2172147?src}. While the intentions might be good, unreflected model set-up and use may have serious consequences.

Some publications to check out:
\begin{itemize}
    \item \url{https://www.sciencedirect.com/science/article/pii/S2214629621003133?via%3Dihub}
    \item \url{https://www.sciencedirect.com/science/article/pii/S001632872300099X?via%3Dihub}
    \item \url{https://www.sciencedirect.com/science/article/pii/S0959652623017432?via%3Dihub}
    \item \url{https://link.springer.com/article/10.1007/s43253-023-00098-7}
    \item \url{https://wires.onlinelibrary.wiley.com/doi/10.1002/wcc.838} I would argue that these feasibility spaces are going in the wrong direction because feasibility, as a subset of plausibility, is entirely subjective. Opening perspectives through the lense of possibility might be more fruitful, especially when considering a plurality of modelling approached and schools of thought. Further, feasibility is constrained to only myopic actions, i.e. what is feasible right now, as everything more distant in the future might appear from the current standpoint as infeasible. Feasibility might be a good guiding light for necessary short-term changes but inadequate for large-scale transformation happening over a long period.
    \item \url{https://arxiv.org/abs/2301.08135}
    \item \url{https://www.nature.com/articles/nclimate2980}
    \item \url{https://www.sciencedirect.com/science/article/pii/S1877343513000808?via%3Dihub}
    \item \url{https://www.sciencedirect.com/science/article/pii/S2214629620304825?via%3Dihub}
    \item \url{https://www.nature.com/articles/s41558-023-01681-w}
    \item \url{https://www.sciencedirect.com/science/article/abs/pii/S2590332223000891}

    \item \url{https://www.emerald.com/insight/content/doi/10.1108/14636680610668045/full/html}
    \item \url{https://www.sciencedirect.com/science/article/pii/S0016328718303264}
    \item maybe a good paper on the role of SDGs in IAMs by van Vuuren \url{https://www.sciencedirect.com/science/article/pii/S2590332222000033?via%253Dihub}
\end{itemize}


On IO-IAM integration, this recent paper is a good summary \url{https://doi.org/10.1080/09535314.2023.2266559}. Not sure if the DYNERIO model was published later \url{https://www.sciencedirect.com/science/article/pii/S092134492300037X}. Also this soft-linked IAM-IO paper by Ju et al. might have been published later \url{https://www.tandfonline.com/doi/abs/10.1080/09535314.2023.2216355}

add some notes on different types of scenarios, e.g. on comparative-static vs dynamic etc., which depends of course in part on the potential/ ability of the model to be used. This might be of interest because feasibility is much less of an important criterion (also because it cannot be really assessed/quantified) in the case of what-if scenarios for only a single point in time. Any model should however be able to reproduce past developments. Even for IO, this is not straightforward when it comes to impact analysis - imputation analysis is something very different. I wonder how that plays out for more complex models, especially IAMs.

In terms of plausibility and possibility, I have to stress the role of statistical models used for scenario analysis (and their only limited suitability).

The examples in the box of the WIRES paper are I think ill-suited because they do not take into account the aspect of time - WHEN! will something happen or someone do something?

Desirability is another aspect to be considered, not necessarily linked directly to either of the terms possibility, plausibility, probability (and even feasibility).

Research integrity also includes that researchers/modellers consider more than the carbon tunnel vision. That is, they must highlight that their analysis focuses mainly on climate change and that trade-offs of environmental impact are possible and that climate change mitigation and adaptation may come at the cost of exacerbating other environmental and social impacts.

The definition of feasibility at one point in time hinges on what has happened in the years before. There is therefore some contingency - the definition is time-dependent

Especially when complex dynamics are involved (such as trade, any kind of social dynamics), "simple" forecasting or predictions are not appropriate anymore - at least when trying to capture big pictures. For very narrow (type and spatiotemporal) cases, that might be still justified though.

This review on scenario has a short section on scenario adaptation and customisation. It also distinguishes between risk and exploratory scenarios, and addresses the problem of putting numbers on qualitative factors \url{https://www.annualreviews.org/doi/full/10.1146/annurev-environ-102017-030109}

Maybe worth a read? On sustainability transitions \url{https://www.sciencedirect.com/science/article/abs/pii/S004873331200056X}

An old (from 2000) paper on scenario modelling in LCA: \url{https://link.springer.com/content/pdf/10.1007/BF02978555.pdf}

Steubing et al. describe in their "super-structures" LCA paper also the role of guidance for LCA practitioners in the sense that they cannot know all underlying models/scenarios but may be held responsible nevertheless \url{https://doi.org/10.1007/s11367-021-01974-2}

Next to PREMISE, there's also FUTURA that allows scenario-based LCAs, although (it seems like) only via manual changes or a simplified batch update (IEA data example). Maybe this is actually a plus of the approach because it requires the modeller to still review the assumptions and data in a detailed manner? \url{https://onlinelibrary.wiley.com/doi/10.1111/jiec.13115}

Ventura argues that conventional (incl. prospective) LCAs are limited by looking only at single functional units and therefore proposes "transition LCA", where an "existing situation" is to be modelled, e.g. substitution of a geographic region  \url{https://www.frontiersin.org/articles/10.3389/frsus.2022.801668/full}. To be integrated with MFA, and distinguished from ALCA and CLCA. (Only skimming the paper, I don't think it is of high quality)

Although based on "expert assumptions", the approach by Douziech et al. is more generic: they combined parameterised foreground and background LCIs to see how background changes affect the foreground \url{https://onlinelibrary.wiley.com/doi/10.1111/jiec.13432} They suggest that non-parameterised background LCIs could be informed by IAMs.

Bisinella et al. review existing future-oriented LCAs and set up very useful archetypes of LCA scenario models; basically, where does the scenario aspect come into play, e.g. a scenario exercise on top of an LCA or doing an LCA using scenario data \url{https://doi.org/10.1007/s11367-021-01954-6} They also advocate for a proper documentation of the scenario developemnt process. The same author also has a paper on prospective waste LCA \url{https://www.sciencedirect.com/science/article/pii/S0956053X23006967}

Arvidsson et al. recently published this paper on terminology for prospective/ ex-ante LCA, arguing that the field should settle on the terms, with them favouring "prospective LCA" \url{https://link.springer.com/article/10.1007/s11367-023-02265-8}

Maybe a useful reference article on comparing LCA methods and analyses by Heijungs \url{https://doi.org/10.1007/s11367-022-02075-4}. Maybe use this reference to highlight that there already large differences in retrospective assessmnts within the same model family

Steubing et al. remind of research integrity and outline an LCA scenario process; they also stress the consistency of scenario/model features although suggest to be pragmatic as well \url{https://link.springer.com/article/10.1007/s11367-023-02192-8}

Refer also to Stefan's book chapter on IAM-IE integration for prospective modelling. It is very similar to the Nature paper \url{https://www.nature.com/articles/nclimate3148}

Pehl et al. and Arvesen et al. are probably still one of the best references concerning prospective IAM-LCA integration \url{https://www.nature.com/articles/s41560-017-0032-9.} and \url{https://www.sciencedirect.com/science/article/pii/S1364815216304005}: THEMIS + REMIND

Another IAM-informed DMFA study, by Pedneault et al., for aluminium and falling back to SSPs \url{https://onlinelibrary.wiley.com/doi/full/10.1111/jiec.13321}

Kermeli et al. use stock-models to inform IAMs, focusing on iron and steel \url{https://www.sciencedirect.com/science/article/pii/S0360544221026839}

Echo the introductory statement in abstract by Wiek et al.:" Sufficient plausibility, in this article, refers to scenarios that hold enough evidence to be considered ‘occurrable’. This might have been the underlying idea of scenarios all along without being explicitly elaborated in a pragmatic concept or methodology." \url{https://www.researchgate.net/profile/Lauren-Keeler/publication/263276841_Plausibility_indications_in_future_scenarios/links/573b616508ae9ace840ea713/Plausibility-indications-in-future-scenarios.pdf}

A review by Stoddard et al. on why global emissions curve is not bent, referring back to IAMs etc. \url{https://www.annualreviews.org/doi/full/10.1146/annurev-environ-012220-011104} 

I guess this LCA overview paper may be relevant \url{https://www.frontiersin.org/articles/10.3389/frsus.2022.629653/full}

Check out this PNAS feature on sustainability transitions of production-consumption systems \url{https://www.pnas.org/topic/551}

Make it also a key point that sustainability scenarios have purposes/targets and that targets of different scenario sets can be conflicting, just like the SDGs have conflicting targets. In that sense, good understanding of the research objective, the used data and models is needed.

Can I come up with a differentiation of when selected variables/parameters of (input/output) scenario data may be used and when not, such that consistency of assumptions (and model/scenario characteristics) is still preserved. E.g. when only population and GDP estimates of any SSP are taken over (while explicitly stating that only this is done) vs when also heated floor space estimates are taken over which contrast regionally/temporally with underlying secondary model assumptions...

Perhaps distinguish between hard-linking and soft-linking (and selected extraction of data)?

Lazurko's PhD thesis on ambiguity and reflexivity in scenarios and sustainability science: \url{https://www.researchgate.net/publication/376273118_Complexity_ambiguity_and_the_boundaries_of_the_future_Toward_a_reflexive_scenario_practice_in_sustainability_science}

Similar to how Boström et al. describe in their introduction (and probably elsewhere, too) for environmental sociology, we should not advocate an "empty" call for more reflexivity in quantitative scenario modelling (regardless of which side of primary or secondary models). \url{https://www.tandfonline.com/doi/full/10.1080/23251042.2016.1237336} That is, more reflexivity is not always better (and this call in itself might be unreflexive).

In prospective IE literature, it appears that scenarios are preactively commensurated, i.e. made comparable 

Van Vuuren's scenario families are a form of scenario categorisation - something that Metz and Hartley say they have not observed in the scenario development literature. Chapter 2.4.2 in \url{https://www.sciencedirect.com/science/article/pii/S004016251930993X?via%3Dihub#sec0014}

Accounting (secondary) models can be used to quantify/describe unobservable structures (based on assumptions, beliefs etc., e.g. attribution to final consumer), thus making them attractive as a source for "indirect"/unobservable data/indicators for use in primary models.

Marti and Gond on performativity, i.e. theories becoming self-fulfilling prophecies (in the corporate/economics context) \url{https://journals.aom.org/doi/10.5465/amr.2016.0071}. Possibly something similar can be said of influential global scenario modelling work, especially the role of the IPCC?

Nilsson et al. on the interactions of SDGs \url{https://www.nature.com/articles/534320a}, and on their conflicting nature: Hickel \url{https://onlinelibrary.wiley.com/doi/full/10.1002/sd.1947}

Part of the adoption of secondary data into primary models may be because of IAM's continued downscaling. IAMs are, however, rather limited in their native representation of society's biophysical basis

"any scenario process produces a partial frame of the future that focuses attention on what is deemed most relevant and is contingent on how it is produced. Without reflexivity, scenario users are left without the means or motivation to critically reflect on the influence of subjective choices made in the design of a scenario development process (e.g., choice of framing or methods) and the strengths and limitations of these choices for their mode of application." on p. 6 of Lazurko's PhD thesis

When arguing for more possible futures: Could it be a problem that some modelling framework are too rigid or based on certain assumptions such that they are unfit to explore such a broader scope? E.g. are LCAs sensible for any kind of co-developed background LCI (and according narrative)?

It would be incorrect to portray secondary models as only "users" of primary scenarios, as there can be more of an interplay ro even cross-fertilisation in some cases

A need for opening up possible futures in secondary modelling (currently shaped mainly by authoritative scenario sets constructed via primary models); reflexivity practice and use of qualifiers may help.

Worth mentioning? Lessons from AR6 for AR7 \url{https://www.nature.com/articles/s44168-023-00082-1}

Does Uruena basically support the point on no-playground when they argue for "foresight as a subject of responsibility"? \url{https://www.sciencedirect.com/science/article/pii/S0016328721001610}

\end{document}